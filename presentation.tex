% Created 2015-03-04 Wed 12:52
\documentclass[big]{beamer}
\usepackage[utf8]{inputenc}
\usepackage[T1]{fontenc}
\usepackage{fixltx2e}
\usepackage{graphicx}
\usepackage{longtable}
\usepackage{float}
\usepackage{wrapfig}
\usepackage{rotating}
\usepackage[normalem]{ulem}
\usepackage{amsmath}
\usepackage{textcomp}
\usepackage{marvosym}
\usepackage{wasysym}
\usepackage{amssymb}
\usepackage{hyperref}
\tolerance=1000
\usepackage{lmodern}
\usetheme{Boadilla}
\usecolortheme{whale}
\setbeamertemplate{footline}{}
\setbeamertemplate{navigation symbols}{}
\setbeamertemplate{itemize items}[default]
\setbeamertemplate{enumerate items}[circle]
\setbeamertemplate{alert}{\textbf}
\usetheme{default}
\author{Matthieu Bruneaux}
\date{2015-03-10}
\title{RAD-seq in Roscoff}
\hypersetup{
  pdfkeywords={},
  pdfsubject={},
  pdfcreator={Emacs 24.3.50.1 (Org mode 8.2.3a)}}
\begin{document}

\maketitle

\section{General introduction to the workshop}
\label{sec-1}

\begin{frame}[label=sec-1-1]{Mini-workshop about ddRAD}
\begin{block}{Introduction about RAD-seq}
\begin{itemize}
\item RAD? RAD-seq? ddRAD?
\item Applications
\item Workflow
\end{itemize}
\end{block}
\begin{block}{Practicals}
\begin{itemize}
\item One complete project, from raw reads to final results
\item Cherry-picking of some analysis steps
\item Open questions
\end{itemize}
\end{block}
\begin{block}{Objectives}
\begin{itemize}
\item Overview of RAD-seq
\item Arouse curiosity
\item Give useful pointers
\end{itemize}
\end{block}
\end{frame}
\begin{frame}[label=sec-1-2]{Disclaimer about the speaker}
\begin{block}{}
\begin{itemize}
\item Not a population geneticist, not a bioinformatician
\item Evolutionary biologist who dropped into a RAD-seq project when he was
post-doc
\item Some things are probably wrong!
\end{itemize}
\end{block}
\end{frame}
\section{Introduction: RAD-seq and population genomics}
\label{sec-2}

\begin{frame}[label=sec-2-1]{What are RAD markers?}
Miller et al. 2007
restriction site polymorphism
\end{frame}
\begin{frame}[label=sec-2-2]{RAD-seq}
\end{frame}

\begin{frame}[label=sec-2-3]{Single read vs. paired ends}
Examples, applications
\end{frame}
\begin{frame}[label=sec-2-4]{ddRAD}
Examples, applications
(population genomics, mapping, QTL, phylogeography, \ldots{})
or put those appplications in more general context?
nonmodel species, marker discovery
anything that needs lots of markers in lots of individuals or pools
\end{frame}
\begin{frame}[label=sec-2-5]{Other flavours}
\end{frame}

\begin{frame}[label=sec-2-6]{Typical analysis}
Depends on the question and objectives!
Experimental/sampling design
DNA extraction, library preps
Barcoding? pooling?
Sequencing (service?)
Reads cleaning
demultiplexing
assembly or alignment (mapping if reference available)
genotype calling
RAD stops here (i.e. lots of markers and allele frequencies for different populations or genotypes for individuals)
Downstream analysis:
\begin{itemize}
\item genome scan
\item Fst, Gst, outlier detection
\item Phylogeography
\item Parallel evolution
\item and many other things\ldots{}
\end{itemize}
\end{frame}
\section{Practicals}
\label{sec-3}

\begin{frame}[label=sec-3-1]{General workflow scheme}
\end{frame}

\begin{frame}[label=sec-3-2]{One complete project}
\end{frame}

\begin{frame}[label=sec-3-3]{Tour of other tools and specific analyses}
To illustrate some specific points (e.g. likelihood or bayesian based genotyping
or allele frequency estimates or Fst calculations, \ldots{})
\end{frame}
% Emacs 24.3.50.1 (Org mode 8.2.3a)
\end{document}
